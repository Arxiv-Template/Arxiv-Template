\begin{abstract}
The arXiv preprint server has become a cornerstone of scientific communication in fields ranging from physics to computer science. However, the lack of a standardized, visually appealing template often results in inconsistent formatting and diminished readability. This paper introduces a new LaTeX-based template designed to address these issues by providing researchers with a clean, unified, and aesthetically pleasing framework for their submissions. Our template emphasizes simplicity, ease of use, and adaptability, ensuring compatibility with diverse research topics while maintaining professional presentation standards. In collaboration with AI models Qwen and DeepSeek, we developed advanced features such as automated layout optimization and semantic content structuring. We demonstrate the effectiveness of our template through qualitative evaluations and user feedback, highlighting its potential to enhance the overall quality of arXiv submissions.

\vspace{2mm}

\textit{Keywords: arXiv, Template, Qwen, DeepSeek}

\vspace{5mm}

\coloremojicode{1F4C5} \textbf{Date}: April 24, 2025

\coloremojicode{1F3E0} \textbf{Projects}: \href{https://wangrongsheng.github.io}{https://wangrongsheng.github.io}

\github{} \textbf{Code Repository}: \href{https://github.com/wangrongsheng}{https://github.com/wangrongsheng}

\wnb{} \textbf{Training Logs}: \href{https://wandb.ai/wangrongsheng}{https://wandb.ai/wangrongsheng}

\coloremojicode{1F917} \textbf{Model Weights \& Checkpoints}: \href{https://huggingface.co/wangrongsheng}{https://huggingface.co/wangrongsheng}

\coloremojicode{1F4DA} \textbf{Datasets}: \href{https://huggingface.co/wangrongsheng}{https://huggingface.co/wangrongsheng}

\coloremojicode{1F4E7} \textbf{Contact}: \href{mailto:wang.rongsheng@outlook.com}{wang.rongsheng@outlook.com}

\end{abstract}
